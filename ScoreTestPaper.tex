%% Paper implementing score test framework for SCR :)

\documentclass{article}
\usepackage{geometry}
\usepackage{layout}
%\geometry{papersize={297mm, 490mm}, left=15mm, right=15mm, top=5mm, bottom=5mm,
%  headheight=5mm, marginpar=5mm}
\usepackage{blindtext}
\usepackage{amsmath}
\usepackage[dvipsnames]{xcolor}
\usepackage{multicol}
%\usepackage{subfig}
\usepackage{caption}
\usepackage{subcaption}
\usepackage{graphicx}
\usepackage{amsfonts}
\usepackage[parfill]{parskip} % Want line break between paragraphs instead of indentation
\usepackage{setspace} % For line spacing
\usepackage{lineno,xcolor} % For numbering lines

%% Bibliography, reference
\usepackage[backend=biber,citestyle=authoryear,uniquename=false,maxcitenames=2,
mincitenames=1,uniquelist=false, style=apa,giveninits=true]{biblatex} % For bibliography
\addbibresource{../../../../../PhDReferences_Zotero.bib} % Bibliography

%% Referencing stuff
\makeatletter
\newcommand{\apamaxcitenames}{2}

 \oddsidemargin  -10mm
 \evensidemargin -10mm
 \headheight -4mm
 \headsep -3mm
\textheight 250mm
\textwidth 180mm
\topmargin -4mm
\topskip -10mm

\title{Using score tests for model selection in spatial capture-recapture}

\begin{document}

\maketitle

% Setting double spacing
\doublespacing
% Adding line numbering
\linenumbers

\section{Introduction}
\label{sec:introduction}

Model selection is an extremely valuable and important step for any
statistical analysis, allowing the practitioner to confirm their
chosen model and any resulting conclusions are likely
reliable. However, it is also a potentially fraught step of
statistical analyses, especially when dealing with
models that require substantial resources and time to fit, such as
spatial capture-recapture models. In such instances, fitting
every candidate model and carrying out model selection using
likelihood ratio tests (LRTs) or comparing AIC values, for example, can be
impractical and unrealistic. A lack of viable alternatives when working with
non-traditional data sets and sophisticated models mean these
traditional model selection methods are often the only option, compromising the practitioner's
ability to confidently select an appropriate model for their analysis.


\end{document}

%%% Local Variables:
%%% mode: latex
%%% TeX-master: t
%%% End:
