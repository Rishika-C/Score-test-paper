%% Paper implementing score test framework for SCR :)

\documentclass{article}
\usepackage{geometry}
\usepackage{layout}
%\geometry{papersize={297mm, 490mm}, left=15mm, right=15mm, top=5mm, bottom=5mm,
%  headheight=5mm, marginpar=5mm}
\usepackage{blindtext}
\usepackage{amsmath}
\usepackage[dvipsnames]{xcolor}
\usepackage{multicol}
%\usepackage{subfig}
\usepackage{caption}
\usepackage{subcaption}
\usepackage{graphicx}
\usepackage{amsfonts}
\usepackage[parfill]{parskip} % Want line break between paragraphs instead of indentation
\usepackage{setspace} % For line spacing
\usepackage{lineno,xcolor} % For numbering lines

%% Bibliography, reference
\usepackage[backend=biber,citestyle=authoryear,uniquename=false,maxcitenames=2,
mincitenames=1,uniquelist=false, style=apa,giveninits=true]{biblatex} % For bibliography
\addbibresource{../../../../../PhDReferences_Zotero.bib} % Bibliography

%% Referencing stuff
\makeatletter
\newcommand{\apamaxcitenames}{2}

 \oddsidemargin  -10mm
 \evensidemargin -10mm
 \headheight -4mm
 \headsep -3mm
\textheight 250mm
\textwidth 180mm
\topmargin -4mm
\topskip -10mm

\title{Using score tests for model selection in spatial capture-recapture}

\begin{document}

\maketitle

% Setting double spacing
\doublespacing
% Adding line numbering
\linenumbers

\section{Introduction}
\label{sec:introduction}

Model selection is an extremely valuable and important step for any
statistical analysis, allowing the practitioner to confirm the
reliability of their chosen model and any resulting
conclusions. However, it is also a potentially fraught step of 
statistical analyses, especially when dealing with sophisticated
models that require substantial resources and time to fit. In such
instances, fitting every candidate model and carrying 
out model selection using likelihood ratio tests (LRTs) or comparing
AIC values, for example, can be impractical and unrealistic. An
absence of viable alternatives when working with non-traditional data
sets and elaborate models mean these traditional model selection
methods are often the only option, compromising the practitioner’s
ability to confidently select an appropriate model for their analysis.

To address this issue, \textcite{catchpoleMorgan1996} propose a
framework harnessing score tests for model selection, greatly reducing
the number of candidate models that need to be fitted 
to select an appropriate model for the data. To date, it has
been applied extensively to the field of capture-recapture, e.g. \textcite{mccreaMorgan2011},
\textcite{mccreaMorganBreg2012} and \textcite{mccreaMorganGim2017}. 

To implement the score test framework, the candidate set is organised into levels
\parencite{mccreaMorgan2011} and, beginning with the simplest
models, score tests define a path through these levels to
the final model. We begin by identifying the
most basic model in the set -- this model 
makes up Level 0. Level 1 contains all models with one added parameter dependency relative
to the Level 0 model. We begin with the null hypothesis
that the Level 0 model is \textbf{adequate} for the data. Our
alternative hypotheses are that each Level 1 model is
\textbf{adequate}. We conduct score tests comparing this null
hypotheses to each alternative hypothesis. The alternative hypothesis
(Level 1 model) for the test with the smallest significant
p-value defines the new null hypothesis. Level 2 is then constructed,
consisting of all models with one additional parameter dependency
compared to this new null model, and the process is repeated. We stop
when none of the score tests produce significant p-values. At this
point, the null hypothesis specifies the final chosen model.

As we discuss later, to calculate the score test statistics, we
use estimates from the null hypothesis models. However, we only
require gradients -- specifically the score vector and information
matrix -- from the models specified under each alternative hypothesis. No
estimates are needed. So at
each step of the process, only the models specified under each null
hypothesis need to be fitted. Furthermore, the stepwise nature of the
framework enhances the convenience of any model-fitting -- the Maximum
Likelihood Estimates (MLEs) from a chosen null model may provide
appropriate starting values for fitting the next null model
\parencite{mccreaMorgan2011}. The issue of choosing a model from a 
potentially vast candidate set of complicated models is reduced to the
task of calculating score statistics and fitting a few, potentially
relatively simple, models.

However, despite these benefits, the application of the framework has
not extended beyond capture-recapture models. The reason is likely
two-fold. Firstly, depending on the likelihood equations for the models
being considered, it may seem prohibitively inconvenient to
obtain the required score vectors and information matrices via
numerical approximations or exact calculations. Previous infrastructure
likely made model-fitting, and therefore traditional model selection
techniques, appear more convenient. Secondly, depending on the
nature of the models being appraised, it may be unclear how to
organise models into levels as described above. There may be a lack of
a clear nesting structure for certain groups of models in the
candidate set.

For example, although it appears a natural extension to previous capture-recapture
applications, the score test framework has not yet been applied to
spatial capture-recapture models. Likelihood equations for these
models can be quite sophisticated -- any gradients cannot be found
symbolically, statistical software must be employed. Additionally,
not all spatial capture-recapture
models can be organised into a clear nested structure. Depending on
the parameter dependencies present in the candidate set, it may not be possible
to build the model levels as described above. \textbf{Such issues are
common} and pose significant obstacles to the widespread use of this
seemingly convenient and helpful approach for model selection.

In this paper, we address these issues. We investigate the use of the
R package `RTMB' to conveniently obtain exact gradients for all models
being considered when implementing the score test framework. We also
propose an extension to the framework, allowing for its use when the
candidate set lacks a clear nested structure. We present these
novel findings via an illustrative example -- we will implement the
score test framework using the ovenbird spatial capture-recapture data
set, available in the `secr' package.

As with the potential difficulties of model selection, our insights
are not specific to any one field of research. They may be 
extended and applied to a vast range of models. Our findings aid in
enhancing the utility of the score test framework, and therefore
enhance the accessibility of model selection, in general. 



\end{document}

%%% Local Variables:
%%% mode: latex
%%% TeX-master: t
%%% End:
